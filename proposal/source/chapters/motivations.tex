\section{Motivations}
Text classification is an important and classical problem in natural language processing. \newline
The goal in the text classification task is to build systems which are able to automatically classify any kind of documents performing different operations such as document organization, news filtering, spam detection, opinion mining, and computational phenotyping. The feature space, based on the set of unique words in the documents, is typically of very high dimension, and thus document classification is not trivial. Currently much research has been made about classical text classification methods such Recurrent Neural Network (RNN), Convolution Neural Network (CNN) and many libraries such FastTex , instead, only a limited number of studies have explored the more flexible Graph Convolutional Neural Networks (GCNN).\\
In this paper the goal is to compare the effectiveness of classical techniques for text classification with respect to Graph Neural Networks (GNN). Comparison will be done  by using different datasets, different tran-validation dataset-split and so on.
The aim of this experiment is to explore the potential of the Graph Neural Networks compared with classical techniques like : Rocchio Classifier, Support Vector Machine, Recurrent Neural Network, Word Vectors, etc., and, find out if the Graph solution has better performance than the others in this particular scenario.
