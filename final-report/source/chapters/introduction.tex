\section{Introduction}
\textbf{Text classification} is an important and classical problem in \textbf{Natural Language Processing}. \newline
The goal in Text Classification task is to build systems which are able to automatically classify any kind of documents performing different operations such as \textbf{document organization}, \textbf{news filtering}, \textbf{spam detection}, \textbf{opinion mining}, and \textbf{computational phenotyping}.
\newline
The \textbf{Feature Space}, based on the set of unique words in the documents, is typically of very high dimension, and thus document classification is not trivial. 
\newline
In the past few years, a lot of research, and progress, has been made, thanks to Text Classification Methods such as \textbf{Recurrent Neural Network (RNN)}, \textbf{Convolution Neural Network (CNN)} and with libraries like \textbf{FastTex}, while, only a limited number of studies have explored the more flexible \textbf{Graph Convolutional Neural Networks (GCNN)}~\cite{paper-graph-convolution-network}.
\newline
In this paper the goal is to compare the effectiveness of classical techniques for \textbf{Text Classification} with respect to \textbf{Graph Neural Networks (GNN)}. Comparisons, will be done,  by using different datasets as well as different tran-validation dataset-split. The aim of this experiment is to explore the potential of the \textbf{Graph Neural Networks} compared with classical techniques~\cite{paper-text-classification-algorithms} like : \textbf{Rocchio Classifier}, \textbf{Support Vector Machine}, \textbf{Recurrent Neural Network}, \textbf{Recurrent Convolutional Neural Network}, etc., and, find out if using a \textbf{Graph Neural Network} solution is better, in terms of performances, than the other methods in this particular scenario .